\noindent\textbf{CONCLUSÃO}
$\!$\\


% Falar sobre o projeto em si
Nesta dissertação, foram apresentados todos os aspectos de hardware, software e econômicos em seus capítulos. Cada um explicando o planejamento a a forma de implementação da ideia, de modo a deixar claro ao leitor como construir e utilizar o sistema como um todo.

O sistema segue sua proposta de escopo aberto, escalável e de baixo custo. Foram apresentadas plataformas de hardware aberto, oferecendo a possibilidade de uma organização ou empresa distribuírem suas próprias versões e baratear ainda mais os custos em larga escala. Os softwares são de escopo aberto e licenças altamente permissíveis, o que significa que possuem versões gratuitas e escaláveis, de modo a também oferecerem a possibilidade de criar versões personalizadas. E por último as versões prontas para uso são de baixo custo, como comprovado em cotações feitas em \ref{section:residencial} e \ref{section:posto}.

A escolha de começar com aplicações feitas com base no TCP/IP, tornou o software flexível para a implementação de novos tipo de protocolos desta camada. Unido a grande quantidade de bibliotecas feitas em diversas linguagens de programação, o sistema pode ser escalado facilmente para aplicações específicas. O modelo Publish/Subscribe é um modelo que reúne características ideais para protocolos full-duplex, em transferências em tempo real, oferecendo a funcionalidade de cada dispositivo escolher quais dados desejam receber através do sistema de tópicos, oferecendo economia energética e de dados.

% Falar das validações dos custos
No capítulo \ref{chapter:casos_de_uso}, foram simulados dois casos de uso, para aplicações residenciais e na indústria/comércio, baseado em preços de mercado, não considerando os preços em atacado que podem baratear o custo para um empresa que deseja vender uma versão desse sistema. O custo total mostrou-se parecido nos dois casos, o que mostra que o sistema é flexível para implementações nos dois tipos de clientes, por outro lado, como a indústria possui uma movimentação monetária e mais recursos para investimento, pode ser mais vantajoso aplicações industrias o do que domésticas, já que o preço é razoavelmente constante.

% Falar sobre intregação com clouds
Vale ressaltar sobre plataformas em nuvem que fornecem serviços IaaS, que podem ser inseridos nas camadas de IoT. Serviços como Brokers, Bancos de Dado, Autenticação, segurança, Inteligência Artificial, Análise e Visualização de dados, estão cada vez mais frequentes no universo do IoT. Isso facilita a implementação do sistema e podem oferecer soluções mais confiáveis e estáveis, porém isso aumenta o custo do sistema, pois esses serviços são pagos.

Por fim, este documento contempla e instrui o usuário sobre a Indústria 4.0, sobre Internet das Coisa e uma forma flexível de implementação do sistema. Para futuras funcionalidade, pode-se incluir ferramentas de segurança, integração com a cloud, outras implementações para bancos de dados e implementações da API em outras linguagens de programação.

\pagebreak



