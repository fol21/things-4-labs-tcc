\noindent\textbf{CONCLUSÃO}
$\!$\\


% Falar sobre o projeto em si
Neste projeto, foram apresentados todos os aspectos de hardware e software em seus capítulos. Encontrou-se uma certa dificuldade em implementar o Data Stream nas duas arquiteturas apresentadas, de modo a ter o mesmo funcionamento em ambas. As linguagens e plataformas usadas possuem uma grande quantidade de bibliotecas que facilitaram a realização dos códigos que implementam Publishers, Subscribers e Data Streams. 

O sistema segue sua proposta de escopo aberto, escalável e de baixo custo. Foram apresentadas plataformas de hardware aberto, oferecendo a possibilidade de uma organização ou empresa distribuírem suas próprias versões e baratear ainda mais os custos em larga escala. Os softwares são de escopo aberto e licenças permissíveis, o que significa que possuem versões gratuitas e escaláveis, de modo a também oferecerem a possibilidade de criar versões personalizadas.

O projeto englobou conceitos nas áreas de Eletrônica, Telecomunicação e Computação espalhadas pelas camadas de IoT. O aprendizado e a integração de ferramentas nessas áreas, permitiu criar um caso de uso que validou o funcionamento do sistema. A medição de temperatura das CPUs foi uma forma de observar o comportamento do sistema, conforme o aumento do número de dispositivos utilizados, além de gerar conclusões a partir das medições de temperatura. Porém, o diferencial são os Data Streams. Sua implementação permitiu diferentes formas de transmitir dados em uma mesma aplicação de uma forma que a arquitetura Publish/Subscribe não permite, além de mudanças dinâmicas destas formas de transmissão.

Planeja-se dar um maior enfoque na implementação dos Data Streams em versões futuras, criando novos tipos de Data Streams e expandindo o suporte para outras linguagens de programação. O sistema já é compatível com o protocolo de segurança SSL/TLS, porém outras formas de segurança como encriptação das mensagens e suporte para Blockchain estão no planejamento para novas versões do sistema, além da funcionalidade de adicionar mais de um broker, para aplicações que exigem um grande fluxo de dados.


\pagebreak



