\chapter{A Interface e sua ligação com IoT}
\label{chapter:interface_iot}

No capítulo \ref{chapter:industria_4_0_iot}, foi visto as bases para se implementar um projeto de IoT. A área começou a receber fortes investimentos e atenção por volta de 2009 e desde então foram feitas consideráveis implementações utilizando tecnologias e protocolos diferentes. Neste capítulo serão apresentados algumas dessas variações, para fins de comparação e respaldo para importância e objetivo deste projeto.


\section{Algumas tecnologias em IoT}
\label{section:tecnologias_iot}

Estas são algumas tecnologias que satisfazem as condições apresentadas para um sistema IoT, nem todas utilizam o protocolo TCP/IP, mas todas são  capazes de fazer seus dispositivos comunicarem-se em tempo real levando em consideração seus alcances e escalabilidade.

\subsection{RFID}
\label{subsection:rfid}


\subsection{Redes NB}
\label{subsection:nb}

%% Inserir diferentes tecnologias

\subsection{Low Energy Bluetooth}
\label{subsection:bluetooth}


\subsection{TCP/IP}
\label{subsection:tcpip}


\section{A Interface}
\label{section:interface}

Inicialmente, os conceitos e ideias do projeto eram voltados a desenvolver uma interface no qual um desenvolvedor poderia implementar um sistema IoT de ponta a ponta utilizando APIs que direcionariam para um desses protocolos da sessão \ref{section:tecnologias_iot}, porém as diferenças entre os protocolos e as camadas de base, fazem com que esta solução esteja mais distante. Então o foco voltou-se  para tecnologias que tenham base na pilha TCP/IP, por sua vasta implementação nas redes industriais e residências e na Internet.

Neste projeto iremos ver a implementação desta interface para o protocolo MQTT, cuja escolha será justificada adiante. Serão descritas as interfaces para as três camadas, que são de baixo custo, open-source e altamente escaláveis para construir outras aplicações com esta como base.
 

