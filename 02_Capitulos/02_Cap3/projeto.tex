\chapter{O Projeto}
\label{chapter:projeto}

Devido a interação entre dispositivos de aquisição de dados e os de aplicação e armazenamento de dados em diferentes cenários (como descrito em \ref{section:overview}), foi necessário uma implementação de um protocolo de comunicação entre os dispositivos em diferentes linguagens de programação.

A arquitetura Publish/Subscribe define as configurações gerais do sistema, não contempla as mudanças de cenário possíveis. É necessária a adição de configurações dinâmicas que se adaptem as condições impostas pelos cenários, uma interface que varia com as condições de cada par Publisher/Subscriber formado. Para isso foi criado o conceito de Data Stream.

O protocolo consiste em configurações de uma abstração de um canal de envio de dados chamado Data Stream mostrado em \ref{fig:3.1.0/data_stream}, no qual transitam dados em uma determinada velocidade, podendo conter um limite de pacote de dados. Nas pontas desse canal estão os Publishers e Subscribers, descritos na seção \ref{section:publishers_subscribers}. Este conceito permite  que a interface possa reagir a limitações de transmissão, como congestionamento na rede. O desenvolvedor não precisará se preocupar com estes detalhes, somente com as configurações do canal.


\subsection{Data Streams}
\label{section:data_stream}

Um Data Stream é uma interface que permite adicionar configurações de como o dado será enviado pelo Publisher  para lidar com os problemas na transmissão ou problemas de processamento. As configurações podem ser enviadas pelo Subscriber, permitindo um dinamismo caso acontece mudanças em cenários onde um Publisher se encontra. A configuração pode lidar com qualquer aspecto do envio de dados, como o tamanho das mensagens enviadas, ou a taxa de envio ou no próprio formato de mensagem enviado.


Publishers são dispositivos que criam Data Stream  e enviam dados por estes, regulam o processamento dos dados estipulam limites de tamanho de cada pacote de dado e determinam o intervalo de envio de pacotes. O protocolo permite que estes enviem os dados e também permite que outros dispositivos possam passar configurações remotamente para modificar os parâmetros de cada Data Stream, como o intervalo de envio ou outra configuração criada pelo tipo de Data Stream implementado. 

\begin{figure}[h!]
\centering
\includegraphics[width=13cm]{./02_Capitulos/02_Cap3/figures/data_stream}
\caption{O conceito de Data Stream para a abstração do transporte de dados}
\label{fig:3.1.0/data_stream}
\end{figure}

Para criar um Data Stream, basta um Subscriber estar ouvindo um tópico no formato abaixo. E um Publisher publicar neste tópico, como na \ref{fig:3.1.0/data_stream_creation}. O Id corresponde a uma identificação única que pode ser definida pelo desenvolvedor. O \textit{stream\_nome} corresponde ao tipo de Data Stream utilizado.

$$ /\{data\_stream\_id\}/stream:\{stream\_nome\} $$

Quando um Data Stream é criado, um conjunto de configurações determina como o dado será enviado, essas configurações estão contidas nos publishers dependendo da lista de Data Streams que este possui. Os Subscribers podem alterar estas configurações, baseada nas necessidades da aplicação, como problemas de processamento ou congestionamento etc.

\begin{figure}[h!]
\centering
\includegraphics[width=13cm]{./02_Capitulos/02_Cap3/figures/data_stream_creation}
\caption{Um Data Stream é criado a partir do tópico \textit{/003/stream:periodic}}
\label{fig:3.1.0/data_stream_creation}
\end{figure}


Para um Subscriber enviar as alterações nas configurações basta publicar no tópico abaixo quase idêntico ao anterior. As configurações são feitas por uma string JSON \cite{json}, um conjunto de chaves-valor universalmente interpretada por várias linguagens de programação como forma de transporte de objetos de uma classe, conforme ilustra a \ref{fig:3.2.0/pub_sub}.

$$ /\{data\_stream\_id\}/configure/stream:\{stream\_nome\} $$

Existem dois tipos de Data Stream já implementados em qualquer Publisher. Porém o desenvolvedor pode implementar seus próprios Data Stream dependendo da linguagem de programação utilizada:

\begin{itemize}
\item Contínuo: Data Stream padrão sem configurações definidas que publica continuamente dados;
\item Periódico: Publica dados esperando um período T antes de publicar, este período pode ser alterado;
\item Customizaveis: Criados pelo desenvolvedor, com suas próprias configurações.
\end{itemize}


\begin{figure}[h!]
\centering
\includegraphics[width=12cm]{./02_Capitulos/02_Cap3/figures/publisher-subscriber_comm}
\caption{Comunicação entre Publishers e Subscribers por Data Stream}
\label{fig:3.2.0/pub_sub}
\end{figure}


\subsection{Implementação em Plataformas}
\label{subsection:plataformas}

A camada de aquisição apresenta implementação dos Publishers, no qual que enviam os dados, as plataformas possuem unidades de processamento e módulos de rede e se comunicam com sensores, para coletar dados físicos, e atuadores que recebem instruções para a execução de alguma função.

\subsubsection{Embarcados}
\label{subsubsection:embarcados}

Embarcados, são sistemas alimentados por baterias, sem alimentação de rede elétrica, portáveis, econômicos, com sistemas de controles geralmente feitos por micro-controladores ou microprocessadores, podendo contemplar sistemas operacionais simples. Com essa descrição, pode-se imaginar que estes dispositivos possuem processamento, energia e desempenho limitados. Neste caso, foi necessário a criação de uma implementação de interface leve e eficiente.

Foram escolhidas para a implementação em embarcados as plataformas microcontroladas com arquitetura Espressif \cite{espressif}. MCUs(Micro-Controller Units), por contemplar módulos WiFi, sensores internos, enstradas e saídas analógicas e digitais e até Bluetooth (não utilizado na versão atual), mostrados na \ref{fig:3.3.4/esp32-arch}, a arquitetura do ESP32. Pela descrição técnica pode-se ver um poder de processamento maior que um Arduino \cite{arduino}, muito utilizado nessas aplicações e que também é compatível com a interface se adicionado shields WiFi. O ESP utiliza linguagem C++ \cite{c++} para desenvolvimento do Publisher e Data Stream, com framework Arduino, que permite implementação em outras plataformas que utilizam este framework, além de uma firmware customizável e circuito open hardware.

\begin{figure}[h!]
\centering
\includegraphics[width=13cm]{./02_Capitulos/02_Cap3/figures/espressif32-arch}
\caption{A arquitetura do ESP32, retirado de \cite{espressif}}
\label{fig:3.3.4/esp32-arch}
\end{figure}


Também foram implementadas, em Javascript  utilizando Node.js \cite{nodejs}, aplicações do Publisher para embarcados com sistemas operacionais, como Raspberry Pi \cite{raspberry-pi}. Permitindo multi-uso entre as funções de Publisher e Subscriber. Esses consoles possuem, processadores mais potentes, módulos de rede, sistemas operacionais, entradas e saídas digitais entre outros.


\subsubsection{Consoles}
\label{subsubsection:consoles}

Consoles são sistemas que contemplam sistemas operacionais, o que permitem mais liberdade para a implementação da interface. Foi escolhida então, realizar a implementação com Node.js , um interpretador de Javascript que permite criar aplicações fora do browser. Possui extensas bibliotecas para HTTP e MQTT além de pipelines que permitem fácil comunicação de protocolos e transição de dados no mesmo processo.


\begin{figure}[h!]
\centering
\includegraphics[width=10cm]{./02_Capitulos/02_Cap3/figures/os-diagram}
\caption{A arquitetura simplificada de dispositivos com Sistema Operacional}
\label{fig:3.3.4/os-diagram}
\end{figure}


Além disso Node.js é uma ferramenta multi-plataforma, com distribuições para Windows, Linux e MAC, além de versões para embarcados de arquitetura ARM, como o próprio Raspberry Pi. Possui Módulos que permitem acessar processos do sistema operacional como ilustrado na \ref{fig:3.3.4/os-diagram} permitindo acesso a Rede além de informações do próprio sistema. O ambiente permite a implementação com programação orientada a objeto, Publishers, Data Streams, Subscribers, são instâncias de classes mostradas no apêndice em \ref{section:codigos_fonte}, permitindo que a aplicação seja feita em um processo (obs: este processo pode executar outros processos, com algum overhead).  Com isso foram implementadas bibliotecas que constroem e interface sobre o MQTT, no lado Subscriber do sistema.

\section{Arquiteturas e Assíncronismo}
\label{section:arquitetura}

Na seção anterior foi discutido a implementação em Hardware do sistema e as diferenças das tecnologias contempladas. Sistemas embarcados 
possuem muito menos funcionalidades que um console gerido por um sistema operacional, sendo uma das principais o paralelismo de processos, a capacidade de ser multi-tarefas, por isso exigiu-se duas filosofias diferentes de implementação do sistema para os dois tipos de plataformas.

\subsection{Arquitetura síncrona em embarcados}
\label{subsection:embarcados_sinc}

Sistemas embarcados possuem um poder de processamento limitado,apesar da tendência de desenvolver embarcados mais potentes. Estes sistemas são geralmente Micro-controlado, ou seja, com arquiteturas mais simples, geralmente executando instruções de somente um programa compilado para linguagem do MCU e gravado neste. Não há paralelismo, cada instrução é síncrona, ou seja, são executadas uma a uma, a próxima deve esperar a anterior acabar de ser executada pela unidade de processamento.

\begin{figure}[h!]
\centering
\includegraphics[width=12cm]{./02_Capitulos/02_Cap3/figures/sinc_implementation}
\caption{Diagrama simplificado de uma rotina padrão seguida pela implementação em Microcontroladores}
\label{fig:sinc-implementation}
\end{figure}

A \ref{fig:sinc-implementation} simplifica a lógica da implementação em dispositivos síncronos como micro-controladores. No Apêndice deste documento pode-se encontrar as implementações em C++. O programa começa configurando o objeto Publisher (MQTT) com configurações de conexão a rede e o broker, além de outras definições do desenvolvedor. O bloco de loop repete-se indeterminadamente, verificando o estado de conexão da aplicação que passa pelos seguintes estados:

\begin{itemize}
\item INIT: Estado inicial, verifica conexão com rede;
\item NETWORK: Tem conexão com rede, verifica conexão com Broker;
\item BROKER: Tem conexão com Broker, inicia configurações dos Data Streams;
\item READY: Pronto para enviar e receber mensagens !
\end{itemize}

Quando o estado está em READY, a aplicação está pronto para processar mensagens recebidas e publicar, lembrando que está é uma lógica básica, o desenvolvedor pode adicionar outros passos, mas a verificações de estado e as configurações do Publisher são obrigatórias para o funcionamento do Sistema. Repare que toda a lógica é sequencial, síncrona, não há paralelismo na rotina.

\subsection{Processos assíncronos em console}
\label{subsection:consoles_assinc}

Consoles são dispositivos que possuem uma Arquitetura mais complexa, consequentemente mais poder de processamento. Ilustrado na  \ref{fig:3.3.4/os-diagram}, a presença de um sistema operacional gerenciando processos do sistema e de aplicações e informações do Hardware, permitem a execução de múltiplas tarefas em paralelo, o que caracteriza processos assíncronos. O interpretador Node.js possui uma gama de bibliotecas para a implementação de funções assíncronas, permitindo o sistema a explorar essa característica e, diferentemente dos embarcados microcontrolados, executar Threads em paralelo, como receber e enviar mensagens simultaneamente.


\begin{figure}[h!]
\centering
\includegraphics[width=13cm]{./02_Capitulos/02_Cap3/figures/async-implementation}
\caption{Diagrama simplificado de uma rotina assíncrona padrão em Consoles}
\label{fig:async-implementation}
\end{figure}

A \ref{fig:async-implementation} mostra o funcionamento simplificado das aplicações de Publisher e Subscriber em Node.js. A aplicação faz as configurações específicas de cada objeto Publisher ou Subscriber e logo após entra em um Loop de eventos, onde existem eventos de Conexão com rede, com o Broker e de envio/recebimento de mensagens, entre outros. Estes eventos quando são acionados pelo sistema, executam callbacks, funções assíncronas definidas e associadas a um evento, previamente nas configurações. Essas funções assíncronas são gerenciadas internamente pelo Node.js e direcionadas ao loop de eventos.

\section{Indexação de dados e Timestamp}
\label{section:timestamp}

Na seção \ref{section:bancos_IoT} foram discutidos estruturas da organização de dados e o formato de armazenamento de dados como o formato de Documento e séries de tempo. É importante que estes formatos tenham formas eficientes de indexação, de modo a facilitar a busca e a análise de dados.

Em uma aplicação de IoT, que envolve coleta de dados em tempo real, é fundamental, independentemente do formato escolhido, a informação de quando este dado foi colhido (data e hora). Isto permite a análise dos dados ao longo do tempo, conforme forem armazenados. Aplicações de decisão e análise utilizam ferramentas estatística com base nas ocorrências temporais, projetando previsões e classificação. No projeto atual, foi implementado o MongoDB um banco de dados de documentos. Cada documento é indexado por uma identificação única como mostrado na \ref{fig:document-model}, permitindo também criar relações entre documentos.

\begin{figure}[h!]
\centering
\includegraphics[width=10cm]{./02_Capitulos/02_Cap3/figures/document-model}
\caption{O formato de documento no MongoDB}
\label{fig:document-model}
\end{figure}

O banco também permite adicionar outros parâmetros de indexação definidos pelo desenvolvedor, foi utilizado desta funcionalidade para adicionar um campo de Timestamp, uma informação de data e hora da inserção no banco. Na implementação foi utilizado o formato Unix Timestamp, o número de milisegundos que se passaram desde 1 de Janeiro de 1970 ás 00:00:00 (UTC), a \ref{fig:document-timestamp} ilustra o formato de documento completo usado no sistema, o campo de dados é definido pelo desenvolvedor. 

\begin{figure}[h!]
\centering
\includegraphics[width=8cm]{./02_Capitulos/02_Cap3/figures/document-timestamp}
\caption{Adição de parâmetro de timestamp em milisegundos ao documento}
\label{fig:document-timestamp}
\end{figure}

Vale ressaltar a existência de uma nova geração de Bancos de Dados TimeScale, um formato parecido com o de Documento, porém com indexação feita por Timestamp ao invés de uma chave única. Em destaque o Banco InfluxDB, que possui estrutura LSM-Tree além de ser um banco de séries de tempo, o que o faz ser utilizado cada vez mais em aplicações IoT.