\chapter{Introdução ao Projeto}
\label{chapter:intro}

O cenário atual do desenvolvimento tecnológico encontra-se no meio de uma quarta revolução industrial. Nunca se produziu tantos dados e se utilizou redes como a própria internet para propaga-los. É de se esperar que tanto a academia e diferentes mercados demandem inovações para o compartilhamento desses dados em tempo real ou próximo disso. Aquecendo o mercado que engloba transporte, análise e inteligência de dados.

Este projeto propõe uma interface para comunicação entre as diferentes tecnologias e camadas de rede, de forma que o desenvolvedor só se preocupe em  implementar e configurar uma interface para mapear a melhor opção de ferramentas para a aplicação. O projeto lida com protocolos baseados na pilha TCP/IP, uma unanimidade em redes conectadas a internet. Podendo se estender para outras protocolos de aplicações de escopo fechado. O foco está no protocolo de aplicação MQTT (\textit{Message Queuing Telemetry Transport}) \cite{mqtt}, um protocolo que opera sobre o TCP/IP, leve e extremamente utilizado para compartilhamento dados telemétricos, de estado e de pequenas mensagens. Oferecendo uma API para aquisição, transmissão, recepção e armazenamento de dados telemétricos..

A Internet das Coisas é a rede que permite a conexão e compartilhamento de dados  de dispositivos físicos . Ela é derivada de métodos de comunicação entre máquinas e telemetria. Pode ser dissecada em três camadas de aquisição, comunicação e aplicação  podendo ser implementada utilizando diversos protocolos de comunicação, dependendo da tecnologia disponível. É importante que sistemas IoT sejam projetados de forma a atender a aplicação eficientemente, porém tal tarefa não é fácil nem simples. Este projeto oferece uma interface que permite facilitar tal tarefa.

\section{Internet das Coisas}
\label{section:iot}

"A Internet das Coisas tem o potencial de mudar o mundo. Assim como a Internet fez. Talvez até mais" \cite{ashton:iot}. Uma tradução livre de Rampim \cite{Rampim:iot} da frase de Kevin Ashton, cofundandor do Auto-ID Center, em 1999. Apesar de ser um nome feito somente para chamar atenção, foi a primeira citação da expressão Internet das Coisas, e de lá vingou.

No contexto da Indústria 4.0, encontra-se a internet das coisas ou IoT, responsável por estruturar as aplicações de aquisição, transmissão e armazenamento de dados a serem analisados. Não é uma surpresa que a Internet das Coisas envolva áreas como eletrônica, computação e telecomunicações em um pacote só. De fato as camadas de IoT são mundos diferentes interligados a um propósito: transmitir dados sobre um dispositivo e/ou para um dispositivo em tempo real. Segundo a Cisco IBSG, Cisco Internet Business Solutions Group \cite{cisco:ibsg}, há mais objetos conectados que pessoas no mundo, fazendo com que o ano de 2009 seja considerado o ano de nascimento da IoT.

Pode-se definir IoT como a estrutura que comunica dispositivos em rede, permitindo a transmissão de dados sobre eles em tempo real. Essa estrutura permite a troca de informações sobre um dispositivo, qual seu estado, seu desempenho, suas condições físicas e do ambiente ao seu redor. Mas, para que este ciclo esteja completo são necessárias camadas que desempenham tarefas específicas, para que o dado chegue a quem ou a o que o está esperando.

\section{Visão geral de uma aplicação IoT}
\label{section:overview}

Na \ref{fig:1.1.0/iot_app} ilustrada, temos uma rede de N sensores que enviam dados telemétricos e M atuadores que recebem ordens para executar uma função, todos estão em rede e podem receber e enviar informação em tempo real. O servidor, que pode ser um Broker como será descrito adiante neste trabalho, encaminha os dados (ou mensagens) para o banco de dados. O Banco é utilizado para análise dos dados, o controlador por sua vez envia as mensagens de decisões baseada na análise de dados a serem transmitidas para os atuadores.

\begin{figure}[h!]
\centering
\includegraphics[width=10cm]{./02_Capitulos/02_Cap1/figures/iot_app}
\caption{Um exemplo de aplicação IoT que percorre os problemas a solucionar}
\label{fig:1.1.0/iot_app}
\end{figure}

O sensor 1 está imerso em um ambiente com usas próprias características físicas e de rede, isso ocorre com todos, isto é cada sensor está imerso num cenário próprio, variando de redes com poucos sensores a redes com grande fluxo de dados, sujeito a congestionamento. Assim, seria de grande ajuda que o  sistema se ajustasse aos diferentes cenários.

Este trabalho visa implementar um sistema que utiliza o protocolo de aplicação MQTT, para transmissão de dados em tempo real entre dispositivos). O sistema contempla também persistência de dados utilizando banco de dados MongoDB, com o diferencial de se adaptar a cenários através de canais de dados chamados Data Streams, oferecendo diferentes configurações que podem otimizar o envio de dados em cada cenário.

\begin{figure}[h!]
\centering
\includegraphics[width=10cm]{./02_Capitulos/02_Cap1/figures/iot_app-layers}
\caption{A aplicação dividida em blocos exercendo um papel em uma aplicação IoT}
\label{fig:1.1.0/iot_app-layers}
\end{figure}


Cada bloco é responsável por uma tarefa no sistema IoT, da aquisição de dados a persistência destes, conforme ilustrado na \ref{fig:1.1.0/iot_app-layers}. Para entender melhor cada tarefa, será descrito o projeto e as implementações em cada bloco no sistema.

Inicialmente, os conceitos e ideias do projeto eram voltados a desenvolver uma interface no qual um desenvolvedor poderia implementar um sistema IoT de ponta a ponta utilizando APIs que direcionariam para um desses protocolos da seção \ref{section:tecnologias_iot}, porém as diferenças entre os protocolos e as camadas de base, fazem com que esta solução esteja mais distante. Então o foco voltou-se  para tecnologias que tenham base na pilha TCP/IP, por sua vasta implementação nas redes industriais e residências e na Internet.
\section{Interface para Protocolos de Aplicação}
\label{section:interface_iot}

Neste projeto será visto a implementação desta interface para o protocolo MQTT, entretanto a idéia é estender o interfaceamento com outros protocolos com caracterísitcas favoráveis para uma rede IoT, como mostrado na \ref{fig:2.2.0/camada_abatracao}, assim como apresentar uma estrutura que se traduza aos protocolos sobre o TCP/IP. Algumas características fundamentais podem ser destacadas como fundamentais:

\begin{itemize}

\item Full-Duplex. Capaz de receber e enviar mensagens ao mesmo tempo;
\item Multicast. Capaz de enviar mensagens um ou mais dispositivos simultâneos;
\item Envio de mensagens em tempo real;

\end{itemize}


\begin{figure}[h!]
\centering
\includegraphics[width=12cm]{./02_Capitulos/02_Cap2/figures/camada_abstracao}
\caption{Interface de comunicação. A interface tem seu próprio protocolo que direciona e se comunica a um ou mais protocolos de aplicação}
\label{fig:2.2.0/camada_abatracao}
\end{figure}

