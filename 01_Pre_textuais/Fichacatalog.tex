% Depois de preparar seu trabalho, você deverá enviá-lo para a Biblioteca CTC/B para avaliação do formato e elaboração da Ficha catalográfica.
% Com a ficha pronta (fornecida pela Biblioteca), você poderá alterar este trecho do trabalho em definitivo.
%
% Para este processo, enviei a dissertação em PDF para o email: ctcb.uerj.bdtd@gmail.com (Tratei de todos os detlahes com a Sra. Márcia)
% Qualquer dúvida, veja os contatos da Biblioteca no site da Rede Sirius: http://www.rsirius.uerj.br/
% 


\begin{titlepage}
	\begin{center}
\vfill
\singlespacing
	\vspace*{85mm}
	{CATALOGAÇÃO NA FONTE\\ \vspace{1.5mm}
	UERJ\,/\,REDE SIRIUS\,/\,BIBLIOTECA CTC/B}\\
	\vspace{1.5mm}
	\begin{boxedminipage}{140mm}
	\begin{minipage}{5mm}
		\vspace{-84mm}
		L732
	\end{minipage}
	\hfill
	\raisebox{8.5mm}{
	\begin{minipage}[top]{115mm}
		\vspace*{5mm}

		de Oliveira Lima, Fernando\\
		\phantom{XX}\setTitulo\,/\,\setNomeAluno -- 2018.\\
		\phantom{XX}\pageref{LastPage}\,f.\\
		\phantom{XX}\\
		\phantom{XX}Orientador: Michel Pompeu Tcheou.\\
		\phantom{XX}Orientador: Lisandro Lovisolo.\\
       		\phantom{XX}Projeto de Graduação apresentado à Universidade do Estado do Rio de Janeiro, Faculdade de Engenharia, para obtenção do grau de bacharel em Engenharia Elétrica.\\
		\phantom{XX} Bibliografia: p.43\\
		\phantom{XX} \\
		\phantom{XX}  1. Internet das Coisas. 2. MQTT. 3. Indústria. I. Tcheou, Michel Pompeu. II. Universidade do Estado do Rio de Janeiro. Faculdade de Engenharia. III. Título.
	\end{minipage}}
	\vspace*{5mm}
	\begin{flushright}
	 CDU~621.3
	\end{flushright}
    \vspace{1mm}
	\end{boxedminipage}\\
	\end{center}
%
	Autorizo, apenas para fins acadêmicos e científicos, a reprodução total ou parcial desta dissertação, desde que citada a fonte.\\
	\noindent
	\begin{tabular}{ccc}
	\phantom{XXXXXXXXXXXXXXXXXXXXXXXXXXXXXX}&	 \phantom{XX}	&	\phantom{XXXXXXXXXXXXXXXX}	\\
	\phantom{XXXXXXXXXXXXXXXXXXXXXXXXXXXXXX}&	 \phantom{XX}	&	\phantom{XXXXXXXXXXXXXXXX}	\\
	\cline{1-1}\cline{3-3}
	Assinatura &		&	Data
	\end{tabular}
\end{titlepage} 