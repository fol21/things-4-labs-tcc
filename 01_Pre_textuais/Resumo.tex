\begin{center}
\textbf{RESUMO}
\end{center}

%
% O resumo deve ser organizado em apenas um parágrafo mesmo.
% O número de folha é o número de páginas do PDF -2. Isto ocorre pois na versão final (capa dura) a capa é removida e as duas primeiras páginas são impressas em uma % folha apenas (frente e verso).
%

$\!$\\

\hspace{-1.3cm}\textbf{LIMA}, Fernando \textit{\setTitulo}. 105 f. Dissertação~(Engenharia Elétrica - Sistemas Eletrônicos) - Faculdade de Engenharia, Universidade do Estado do Rio de Janeiro (UERJ), Rio de Janeiro, 2018.

\vspace{.2cm}

No meio da revolução dos dados, cresce o interesse em comunicação entre máquinas e compartilhamento de dados telemétricos sobre dispositivos, seja numa fábrica ou em residências. Esta dissertação trata sobre um sistema para aplicações de internet das coisas(IoT) utilizando MQTT a \textit{lingua franca} para publicação de dados telemétricos via TCP/IP, com persistência de dados em banco MongoDB. Englobando todos os setores de aquisição dos dados a camada de aplicação em consoles, com o objetivo de facilitar a implementação de aplicações eficientes em cada cenário.

\vspace{1cm}

\hspace{-1.3cm}Palavras-chave: iot, mqtt, indústria.