\noindent\textbf{INTRODUÇÃO}
$\!$\\

O cenário atual do desenvolvimento tecnológico encontra-se no meio de uma quarta revolução industrial. Nunca se produziu tantos dados e utilizou-se redes como a própria internet para propaga-los. É de se esperar que tanto o cenário acadêmico e o próprio mercado demandem inovações para o compartilhamento desses dados em tempo real ou próximo disso. Fazendo aquecer o mercado que engloba transporte, análise e inteligência de dados.

Essa revolução possui um nome, Indústria 4.0. Ela engloba todas as áreas que lidam com dados, da análise à rede que distribui os dados. E dentre estas áreas complexas, que envolvem quase todos os subgrupos da engenharia elétrica, encontra-se a Internet das Coisas, ou IoT, como iremos nos referenciar nesta dissertação.

A Internet das Coisas é a rede ou sistema que adquire, compartilha e aplica dados de dispositivos previamente equipados para medir e divulga-los. Ela é derivada métodos de comunicação entre máquinas e telemetria. Pode ser dissecada em três camadas de aquisição, comunicação e aplicação destes dados e pode ser implementada utilizando diversos protocolos de comunicação, dependendo da tecnologia disponível.

Este projeto propõe uma interface para lidar com o a pilha TCP/IP, uma unanimidade em redes que se comunicam com a internet. Podendo se estender para outros protocolos de aplicações de escopo fechado. O foco está no protocolo de aplicação MQTT (Message Queuing Telemetry Transport), um protocolo que trabalho em cima do TCP/IP, leve e extremamente utilizado para compartilhamento dados telemétricos, de estado e de pequenas mensagens. Oferecendo uma API para tanto a aquisição assim como o recebimento e armazenamento destes dados.

 