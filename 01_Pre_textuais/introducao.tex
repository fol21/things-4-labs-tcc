\noindent\textbf{INTRODUÇÃO}
$\!$\\

O cenário atual do desenvolvimento tecnológico encontra-se no meio de uma quarta revolução industrial. Nunca se produziu tantos dados e utilizou-se redes como a própria internet para propaga-los. É de se esperar que tanto o cenário acadêmico e o próprio mercado demandem inovações para o compartilhamento desses dados em tempo real ou próximo disso. Fazendo aquecer o mercado que engloba transporte, análise e inteligência de dados.

Este projeto propõe uma interface que fornece comunicação entre as diferentes tecnologias e camadas de rede, de forma que o usuário só se preocupe em  implementar e configurar a interface para mapear a melhor opção de ferramentas para a aplicação. O projeto lida com protocolos baseados na pilha TCP/IP, uma unanimidade em redes que se comunicam com a internet. Podendo se estender para outras protocolos de aplicações de escopo fechado. O foco está no protocolo de aplicação MQTT (Message Queuing Telemetry Transport), um protocolo que trabalho em cima do TCP/IP, leve e extremamente utilizado para compartilhamento dados telemétricos, de estado e de pequenas mensagens. Oferecendo uma API para tanto a aquisição assim como o recebimento e armazenamento destes dados.

A Internet das Coisas é a rede ou sistema que adquire, compartilha e aplica dados de dispositivos previamente equipados para medir e divulga-los. Ela é derivada métodos de comunicação entre máquinas e telemetria. Pode ser dissecada em três camadas de aquisição, comunicação e aplicação destes dados e pode ser implementada utilizando diversos protocolos de comunicação, dependendo da tecnologia disponível. É importante que o sistemas IoT deva ser construído de forma a atender a aplicação de uma forma eficiente, porém tal tarefa não é fácil nem simples. Este projeto oferece uma interface que permite facilitar esta tarefa.


 