\noindent\textbf{INTRODUÇÃO}
$\!$\\

O cenário atual do desenvolvimento tecnológico encontra-se no meio de uma quarta revolução industrial. Nunca se produziu tantos dados e se utilizou redes como a própria internet para propaga-los. É de se esperar que tanto a academia e diferentes mercados demandem inovações para o compartilhamento desses dados em tempo real ou próximo disso. Aquecendo o mercado que engloba transporte, análise e inteligência de dados.

Este projeto propõe uma interface para comunicação entre as diferentes tecnologias e camadas de rede, de forma que o desenvolvedor só se preocupe em  implementar e configurar uma interface para mapear a melhor opção de ferramentas para a aplicação. O projeto lida com protocolos baseados na pilha TCP/IP, uma unanimidade em redes conectadas a internet. Podendo se estender para outras protocolos de aplicações de escopo fechado. O foco está no protocolo de aplicação MQTT (\textit{Message Queuing Telemetry Transport}) \cite{mqtt}, um protocolo que opera sobre o TCP/IP, leve e extremamente utilizado para compartilhamento dados telemétricos, de estado e de pequenas mensagens. Oferecendo uma API para aquisição, transmissão, recepção e armazenamento de dados telemétricos..

A Internet das Coisas é a rede que permite a conexão e compartilhamento de dados  de dispositivos físicos . Ela é derivada de métodos de comunicação entre máquinas e telemetria. Pode ser dissecada em três camadas de aquisição, comunicação e aplicação  podendo ser implementada utilizando diversos protocolos de comunicação, dependendo da tecnologia disponível. É importante que sistemas IoT sejam projetados de forma a atender a aplicação eficientemente, porém tal tarefa não é fácil nem simples. Este projeto oferece uma interface que permite facilitar tal tarefa.


 