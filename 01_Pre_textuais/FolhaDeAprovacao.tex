\addtocounter{page}{+1}
\begin{center}

\setNomeAluno

\vspace{1cm}

\textbf{\setTitulo}

\end{center}

\vspace{.4cm}

\begin{flushright}
\parbox{8cm}{
\singlespacing{Trabalho de Conclusão de Curso apresentado, como requisito parcial para obtenção do título de Bacharel em Engenharia Elétrica ênfase em Sistemas Eletrônico, da Universidade do Estado do Rio de Janeiro.}
}
\end{flushright}

\vspace{.6cm}


% insira abaixo a data de sua defesa
% Caso não tenha defendido ainda, deixe em branco

\noindent Aprovado em: \setApprovalDate

\noindent Banca Examinadora:


%
%
% Os professores da UERJ DEVEM ser citados primeiro, independente de quem seja o orientador.
%
%



\vspace{.7cm}

\begin{flushright}
\parbox{12cm}{

\singlespacing

\hrulefill \\

\vspace{-.4cm}
Prof. Dr. Michel Pompeu Tcheou (Orientador)
\newline
Departamento de Eletrônica e Telecomunicações da UERJ
\vspace{.7cm}

\hrulefill \\

\vspace{-.4cm}
Prof. Dr. Lisandro Lovisolo (Orientador)
\newline
Departamento de Eletrônica e Telecomunicações da UERJ
\vspace{.7cm}


\hrulefill \\

\vspace{-.4cm}
Prof. Dr. Téo Cerqueira Revoredo
\newline
Departamento de Eletrônica e Telecomunicações da UERJ
\vspace{.7cm}

}

\end{flushright}
\vfill
	

\begin{center}
\setLocationDate
\end{center}
