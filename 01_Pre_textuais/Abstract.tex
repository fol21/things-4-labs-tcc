\begin{center}
\textbf{ABSTRACT}
\end{center}

$\!$\\

\hspace{-1.3cm}\textbf{LIMA}, Fernando \textit{\setTituloEnglish}. \pageref{LastPage} f. Graduation Project~(Electrical Engineering) - Faculdade de Engenharia, Universidade do Estado do Rio de Janeiro (UERJ), Rio de Janeiro, 2018.

\vspace{.2cm}

% O resumo em inglês deve ser organizado em apenas um parágrafo mesmo.

In the verge of the data revolution, a growing interest in communication sytems between machines and the sharing systems of telemetric data on devices rises, whether in a factory or in a residence. This work presents a system for Internet applications of things (IoT) using MQTT, an application protocol for communication between devices that shares telemetric data, with data persistion using MongoDB. The system will  encompass all sectors of data acquisition to the application layer, facilitating implementations of applications in each scenario.

The system also contemplates Data Streams, a configurable data channel, allowing you to define parameters how data will be transmitted. With Data Streams, it is possible for the system to dynamically adapt to scenarios where difficulties in data transmission are found, such as problems in connection stability, network congestion, and so on.

The project also contemplates a use case of the system, where temperature measurements are made in CPUs and in the micro-controlled unit ESP32, making three measurements. The first one being the measurement of an isolated CPU, evaluating its behavior, the second, a comparative between measurements of an Intel CPU and the ESP32 and finally, the comparative of Multiple computers CPUs temperatures in PROSAICO laboratory.

\vspace{1cm}

\hspace{-1.3cm}Keywords: IoT, MQTT, industry .