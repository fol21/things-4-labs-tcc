% Modelo de Dissertação em Latex para o PPG em Engenharia Elétrica - Sistemas Eletrônicos da UERJ
% Este modelo foi adaptado da versão disponibilizada no site da Engenharia Elétrica da UERJ
% http://www.pel.uerj.br/publico/Modelo_LaTeX_Dissertacao_UERJ.rar
% http://www.pel.uerj.br/defesas/
% 
%
% 
%
% 
% Atualizações:
% Felipe M. - 20/06/2012
% Fernando Lima - 15/07/2015
%





\documentclass[a4paper,12pt,oneside,openany]{uerj}

\usepackage[english,brazil]{babel}
%\usepackage[latin1]{inputenc}
\usepackage[utf8]{inputenc}
\usepackage{enumerate}
\usepackage{cite}
\usepackage{epsf,epsfig,./packages/psfig}
\usepackage{./packages/pagina}
\usepackage{indentfirst}
\usepackage{theorem}
\usepackage{fancyhdr}
\usepackage{setspace}
\usepackage{boxedminipage}
\usepackage{float}
\usepackage{makeidx}
\usepackage{amsmath}
\usepackage[hidelinks]{hyperref}
\usepackage{listings}
\usepackage{xcolor}




%%%%%% Definições %%%%%%%
\newtheorem{deff}{Definição}[section]
\numberwithin{equation}{chapter}

\theoremstyle{plain}

\bibliographystyle{./bib/abnt-num}



%%%%% DOC %%%%%%%%%%%%


\makeindex


\begin{document}


%%% Modifique aqui seu Nome, título do trabalho e data

\newcommand{\setNomeAluno}{Fernando de Oliveira Lima}
\newcommand{\setTitulo}{Sistema Escalável para Aplicações de Internet das Coisas utilizando MQTT}
\newcommand{\setLocationDate}{Rio de Janeiro\\2018} % Coloque aqui a localização e data
\newcommand{\setApprovalDate}{8 de Outubro de 2018}


\hypersetup{
    colorlinks,
    citecolor=black,
    filecolor=black,
    linkcolor=black,
    urlcolor=black,
    linktoc=all,
}

\thispagestyle{empty}\input{./01_Pre_textuais/Capa}
\pagebreak\thispagestyle{empty}\begin{center}

\setNomeAluno
% \vfill
\vspace{2cm}

\textbf{\setTitulo}

\vspace{1.0cm}

\begin{figure}[hbt!]
\begin{center}
\includegraphics[width=10.48cm,height=10.8cm]{./01_Pre_textuais/figures/logo_uerj_gnd_pb.png}
\end{center}
\end{figure}

\vspace{-9cm}
\begin{flushright}
\parbox{8cm}{
\singlespacing{Projeto de Graduação apresentado, como requisito parcial  para obtenção do título de Bacharel em Engenharia Elétrica ênfase em Sistemas Eletrônico, da Universidade do Estado do Rio de Janeiro.}
}
\end{flushright}

\vspace{4.0cm}


\begin{table}[h!]
\centering
\begin{tabular}{ll}
Orientadores: & Prof. Michel Tcheou, DSc\\
					& Prof. Lisandro Lovisolo, DSc\\
\end{tabular}
\end{table}


\par\vfill
%\vspace{2cm}

\setLocationDate
\end{center}
\pagebreak\thispagestyle{empty}% Depois de preparar seu trabalho, você deverá enviá-lo para a Biblioteca CTC/B para avaliação do formato e elaboração da Ficha catalográfica.
% Com a ficha pronta (fornecida pela Biblioteca), você poderá alterar este trecho do trabalho em definitivo.
%
% Para este processo, enviei a dissertação em PDF para o email: ctcb.uerj.bdtd@gmail.com (Tratei de todos os detlahes com a Sra. Márcia)
% Qualquer dúvida, veja os contatos da Biblioteca no site da Rede Sirius: http://www.rsirius.uerj.br/
% 


\begin{titlepage}
	\begin{center}
\vfill
\singlespacing
	\vspace*{95mm}
	{CATALOGAÇÃO NA FONTE\\ \vspace{1.5mm}
	UERJ\,/\,REDE SIRIUS\,/\,BIBLIOTECA CTC/B}\\
	\vspace{1.5mm}
	\begin{boxedminipage}{140mm}
	\begin{minipage}{5mm}
		\vspace{-84mm}
		L732
	\end{minipage}
	\hfill
	\raisebox{8.5mm}{
	\begin{minipage}[top]{115mm}
		\vspace*{5mm}

		de Oliveira Lima, Fernando\\
		\phantom{XX}\setTitulo\,/\,\setNomeAluno -- 2018.\\
		\phantom{XX}\pageref{LastPage}\,f.\\
		\phantom{XX}\\
		\phantom{XX}Orientador: Michel Pompeu Tcheou.\\
       		\phantom{XX}Projeto de Graduação apresentado à Universidade do Estado do Rio de Janeiro, Faculdade de Engenharia, para obtenção do grau de bacharel em Engenharia Elétrica.\\
		\phantom{XX} Bibliografia: p.43\\
		\phantom{XX} \\
		\phantom{XX}  1. Internet das Coisas. 2. MQTT. 3. Indústria. I. Tcheou, Michel Pompeu. II. Universidade do Estado do Rio de Janeiro. Faculdade de Engenharia. III. Título.
	\end{minipage}}
	\vspace*{5mm}
	\begin{flushright}
	 CDU~621.3
	\end{flushright}
    \vspace{1mm}
	\end{boxedminipage}\\
	\end{center}
%
	Autorizo, apenas para fins acadêmicos e científicos, a reprodução total ou parcial desta dissertação, desde que citada a fonte.\\
	\noindent
	\begin{tabular}{ccc}
	\phantom{XXXXXXXXXXXXXXXXXXXXXXXXXXXXXX}&	 \phantom{XX}	&	\phantom{XXXXXXXXXXXXXXXX}	\\
	\phantom{XXXXXXXXXXXXXXXXXXXXXXXXXXXXXX}&	 \phantom{XX}	&	\phantom{XXXXXXXXXXXXXXXX}	\\
	\cline{1-1}\cline{3-3}
	Assinatura &		&	Data
	\end{tabular}
\end{titlepage} 
\pagebreak\thispagestyle{empty}\addtocounter{page}{+1}
\begin{center}

\setNomeAluno

\vspace{1cm}

\textbf{\setTitulo}

\end{center}

\vspace{.4cm}

\begin{flushright}
\parbox{8cm}{
\singlespacing{Trabalho de Conclusão de Curso apresentado, como requisito parcial para obtenção do título de Bacharel em Engenharia Elétrica ênfase em Sistemas Eletrônico, da Universidade do Estado do Rio de Janeiro.}
}
\end{flushright}

\vspace{.6cm}


% insira abaixo a data de sua defesa
% Caso não tenha defendido ainda, deixe em branco

\noindent Aprovado em: \setApprovalDate

\noindent Banca Examinadora:


%
%
% Os professores da UERJ DEVEM ser citados primeiro, independente de quem seja o orientador.
%
%



\vspace{.7cm}

\begin{flushright}
\parbox{12cm}{

\singlespacing

\hrulefill \\

\vspace{-.4cm}
Prof. Dr. Michel Tcheou (Orientador)
\newline
Departamento de Eletrônica e Telecomunicações  da UERJ
\vspace{.7cm}

\hrulefill \\

\vspace{-.4cm}
Prof. Dr. Nome do Professor 2
\newline
Faculdade de Engenharia da UERJ
\vspace{.7cm}

\hrulefill \\

\vspace{-.4cm}
Prof. Dr. Nome do Professor 3
\newline
Universidade Federal do Rio de Janeiro - UFRJ - COPPE
\vspace{.7cm}

\hrulefill \\

\vspace{-.4cm}
Prof. Dr. Nome do Professor 4
\newline
Instituto de Geociências da UFF
\vspace{.7cm}

\hrulefill \\

\vspace{-.4cm}
Prof. Dr. Nome do Professor 5
\newline
Universidade Federal do Rio de Janeiro - UFRJ - COPPE
\vspace{.7cm}

}
\end{flushright}
\vfill

\begin{center}
\setLocationDate
\end{center}

\pagebreak\thispagestyle{empty}\begin{center}
\textbf{DEDICATÓRIA}
\end{center}

$\!$\\

%\vspace{1cm}

A todos que tiveram a paciência para ler esse pequeno texto...
\vspace{1cm}

Dedico este trabalho, em primeiro lugar, meus pais e familiares, que vencerem muitas batalhas antes dessa, para que eu esteja aqui, feliz, saudável e com uma condição privilegiada em relação a muitos concluindo mais uma etapa de minha vida. Me ensinaram valores inestimáveis como respeito, educação, compaixão, coragem, determinação entre outros valores não menos importantes. Espero um dia fazer o mesmo por alguém.

Dedico também a todos aqueles que de alguma foram me apoiaram e me ajudaram, seja pelo gesto mais simples, vocês me inspiram, me motivam e sem dúvidas fazem parte de cada palavra deste trabalho.


Não é fácil encontrar sua vocação, o seu lugar. Tive a sorte de definir o que seria de minha vida relativamente cedo e não me arrepender. Mas saiba que existem pessoas que estão tentando de alguma forma se encontrar (ou re-encontrar), seja por problemas pessoas, desilusões ou decepções no passado. Para essas pessoas, fica aqui minha dedicatória, tenha resiliência, sempre vale a pena continuar, uma hora a gente se encontra e logo em seguida já quer outra coisa, faz parte, a insatisfação nos move para frente. Nenhuma decepção é definitiva, tem jeito pra tudo, menos pra morte. Levante a cabeça, ninguém disse que seria fácil, não é ?

\vfill

$SIC * PARVIS * MAGNA$
\pagebreak\thispagestyle{empty}\begin{center}
\textbf{AGRADECIMENTO}
\end{center}

$\!$\\

Agradeço aos amigos e aos não tão amigos, que igualmente fizeram parte da minha evolução como ser humano e cidadão. Todas as pessoas que passaram na minha vida contribuíram um pouco para esse momento, não há como saber onde todas estão, mas se um dia lerem esse texto, quero que saiba que eu agradeço a companhia e o aprendizado.

Não poderia ficar de fora, é claro, todos aqueles que de Alguma forma foram meus mentores. Meus professores ao longo da vida, em especial os dois orientadores desse projeto, já os conheço a quase sete anos, fizeram parte da minha formação, me deram oportunidades e ajudaram a concretizar esse trabalho. Agradeço de coração, que mais oportunidades de nos encontrar surjam, e desejo todo o sucesso possível e uma vida feliz e confortável.

Fica meu agradecimento aqui também para essa instituição linda e maravilhosa, que não é fácil de lidar, muitas vezes pareceu que ia nos abandonar, mas trouxemos ela de volta, cada um com sua parte. Obrigado UERJ, por uma parte inesquecível da minha vida e por todas as pessoas que conheci dentro deste universidade.

E se você estiver lendo esse texto, sim, você mesmo leitor! Agradeço seu prestígio, mesmo se você tiver ignorado completamente esta parte. Espero que tenha aprendido um pouco. O texto é para Ciência, para a Tecnologia e para você.

 
%\pagebreak\thispagestyle{empty}\input{./01_Pre_textuais/Epigrafe}    % não coloquei epígrafe no meu trabalho, mas fica aqui a chamada comentada.
\pagebreak\thispagestyle{empty}\begin{center}
\textbf{RESUMO}
\end{center}

%
% O resumo deve ser organizado em apenas um parágrafo mesmo.
% O número de folha é o número de páginas do PDF -2. Isto ocorre pois na versão final (capa dura) a capa é removida e as duas primeiras páginas são impressas em uma % folha apenas (frente e verso).
%

$\!$\\

\hspace{-1.3cm}\textbf{LIMA}, Fernando \textit{\setTitulo}. 105 f. Trabalho de Conclusão de Curso~(Engenharia Elétrica ênfase em Sistemas Eletrônicos) - Faculdade de Engenharia, Universidade do Estado do Rio de Janeiro (UERJ), Rio de Janeiro, 2018.

\vspace{.2cm}

No meio da revolução dos dados, cresce o interesse em sistemas de comunicação entre máquinas e sistemas de compartilhamento e visualização de dados  sobre dispositivos, seja numa fábrica ou em residências. Este trabalho apresenta um sistema para aplicações de internet das coisas(IoT) utilizando MQTT, um protocolo de aplicação para comunicação entre dispositivos que enviam dados telemétricos. É  a \textit{língua franca} para publicação de dados telemétricos via TCP/IP, com persistência de dados em banco MongoDB. O sistema englobará todos os setores de aquisição dos dados a camada de aplicação em consoles, com o objetivo de facilitar a implementação de aplicações eficientes em cada cenário.

\vspace{1cm}

\hspace{-1.3cm}Keywords: IoT, MQTT, industry .
\pagebreak\thispagestyle{empty}\begin{center}
\textbf{ABSTRACT}
\end{center}

$\!$\\

% O resumo em inglês deve ser organizado em apenas um parágrafo mesmo.

In the verge of the data revolution, a growing interest in communication sytems between machines and the sharing systems of telemetric data on devices rises, whether in a factory or in a residence. This work presents a system for Internet applications of things (IoT) using MQTT, an application protocol for communication between devices that shares telemetri data . It is the \textit{língua franca} for publishing telemetric data via TCP / IP, with data persistion using MongoDB. The system will  encompass all sectors of data acquisition to the application layer in consoles, facilitating implementations of applications in each scenario.

\vspace{1cm}

\hspace{-1.3cm}Keywords: IoT, MQTT, industry .

\fancypagestyle{plain}{
\fancyhf{} % clear all header and footer fields
\renewcommand{\headrulewidth}{0pt}
\renewcommand{\footrulewidth}{0pt}}
\pagestyle{plain}

\pagebreak

\def\listfigurename{LISTA DE FIGURAS}\listoffigures
\def\listtablename{LISTA DE TABELAS}\listoftables

% Caso tenha códigos
\renewcommand{\lstlistingname}{Códigos}
\renewcommand{\lstlistlistingname}{Lista de \lstlistingname}\lstlistoflistings

\newpage

\begin{center}
\textbf{LISTA DE SIGLAS}
\end{center}
$\!$\\

\begin{tabular}{lll}
IoT & \hspace{1cm} & Internet das Coisas \\
MQTT & \hspace{1cm} & Message Queuing Telemetry Transport \\
API & \hspace{1cm} & Application Programming Interface \\
NB &\hspace{1cm} &  Narrow Band Networks \\
A/D & \hspace{1cm} & Analógico-Digital \\
DB &  \hspace{1cm} & Banco de Dados \\
IaaS & \hspace{1cm} & Infrastructure as a Service \\
MCU & \hspace{1cm} & Micro-Controller Unit \\
\end{tabular}

\def\contentsname{SUMÁRIO}\tableofcontents

\fancypagestyle{plain}{
\fancyhf{} % clear all header and footer fields
\fancyhead[R]{\thepage}
\setlength{\voffset}{-1cm}
\setlength{\headsep}{1cm}
\renewcommand{\headrulewidth}{0pt}
\renewcommand{\footrulewidth}{0pt}}

\pagestyle{plain}

\pagebreak
%\addcontentsline{toc}{chapter}{\hspace{1.7cm}\bfseries INTRODUÇÃO}
%\noindent\textbf{INTRODUÇÃO}
$\!$\\

O cenário atual do desenvolvimento tecnológico encontra-se no meio de uma quarta revolução industrial. Nunca se produziu tantos dados e utilizou-se redes como a própria internet para propaga-los. É de se esperar que tanto o cenário acadêmico e o próprio mercado demandem inovações para o compartilhamento desses dados em tempo real ou próximo disso. Fazendo aquecer o mercado que engloba transporte, análise e inteligência de dados.

Essa revolução possui um nome, Indústria 4.0. Ela engloba todas as áreas que lidam com dados, da análise à rede que distribui os dados. E dentre estas áreas complexas, que envolvem quase todos os subgrupos da engenharia elétrica, encontra-se a Internet das Coisas, ou IoT, como iremos nos referenciar nesta dissertação.

A Internet das Coisas é a rede ou sistema que adquire, compartilha e aplica dados de dispositivos previamente equipados para medir e divulga-los. Ela é derivada métodos de comunicação entre máquinas e telemetria. Pode ser dissecada em três camadas de aquisição, comunicação e aplicação destes dados e pode ser implementada utilizando diversos protocolos de comunicação, dependendo da tecnologia disponível. É importante que o sistemas IoT deva ser construído de forma a atender a aplicação de uma forma eficiente, porém tal tarefa não é fácil nem simples.

Este projeto propõe uma interface que fornece comunicação entre as diferentes tecnologias e camadas de rede, de forma que o usuário só se preocupe em  implementar e configurar a interface para mapear a melhor opção de ferramentas para a aplicação. O projeto lida com protocolos baseados na pilha TCP/IP, uma unanimidade em redes que se comunicam com a internet. Podendo se estender para outras protocolos de aplicações de escopo fechado. O foco está no protocolo de aplicação MQTT (Message Queuing Telemetry Transport), um protocolo que trabalho em cima do TCP/IP, leve e extremamente utilizado para compartilhamento dados telemétricos, de estado e de pequenas mensagens. Oferecendo uma API para tanto a aquisição assim como o recebimento e armazenamento destes dados.

 



\chapter{Introdução ao Projeto}
\label{chapter:intro}

O cenário atual do desenvolvimento tecnológico encontra-se no meio de uma quarta revolução industrial. Nunca se produziu tantos dados e se utilizou redes como a própria internet para propaga-los. É de se esperar que tanto a academia e diferentes mercados demandem inovações para o compartilhamento desses dados em tempo real ou próximo disso. Aquecendo o mercado que engloba transporte, análise e inteligência de dados.

Este projeto propõe uma interface para comunicação entre as diferentes tecnologias e camadas de rede, de forma que o desenvolvedor só se preocupe em  implementar e configurar uma interface para mapear a melhor opção de ferramentas para a aplicação. O projeto lida com protocolos baseados na pilha TCP/IP, uma unanimidade em redes conectadas a internet. Podendo se estender para outras protocolos de aplicações de escopo fechado. O foco está no protocolo de aplicação MQTT (\textit{Message Queuing Telemetry Transport}) \cite{mqtt}, um protocolo que opera sobre o TCP/IP, leve e extremamente utilizado para compartilhamento dados telemétricos, de estado e de pequenas mensagens. Oferecendo uma API para aquisição, transmissão, recepção e armazenamento de dados telemétricos..

A Internet das Coisas é a rede que permite a conexão e compartilhamento de dados  de dispositivos físicos . Ela é derivada de métodos de comunicação entre máquinas e telemetria. Pode ser dissecada em três camadas de aquisição, comunicação e aplicação  podendo ser implementada utilizando diversos protocolos de comunicação, dependendo da tecnologia disponível. É importante que sistemas IoT sejam projetados de forma a atender a aplicação eficientemente, porém tal tarefa não é fácil nem simples. Este projeto oferece uma interface que permite facilitar tal tarefa.

\section{Internet das Coisas}
\label{section:iot}

"A Internet das Coisas tem o potencial de mudar o mundo. Assim como a Internet fez. Talvez até mais" \cite{ashton:iot}. Uma tradução livre de Rampim \cite{Rampim:iot} da frase de Kevin Ashton, cofundandor do Auto-ID Center, em 1999. Apesar de ser um nome feito somente para chamar atenção, foi a primeira citação da expressão Internet das Coisas, e de lá vingou.

No contexto da Indústria 4.0, encontra-se a internet das coisas ou IoT, responsável por estruturar as aplicações de aquisição, transmissão e armazenamento de dados a serem analisados. Não é uma surpresa que a Internet das Coisas envolva áreas como eletrônica, computação e telecomunicações em um pacote só. De fato as camadas de IoT são mundos diferentes interligados a um propósito: transmitir dados sobre um dispositivo e/ou para um dispositivo em tempo real. Segundo a Cisco IBSG, Cisco Internet Business Solutions Group \cite{cisco:ibsg}, há mais objetos conectados que pessoas no mundo, fazendo com que o ano de 2009 seja considerado o ano de nascimento da IoT.

Pode-se definir IoT como a estrutura que comunica dispositivos em rede, permitindo a transmissão de dados sobre eles em tempo real. Essa estrutura permite a troca de informações sobre um dispositivo, qual seu estado, seu desempenho, suas condições físicas e do ambiente ao seu redor. Mas, para que este ciclo esteja completo são necessárias camadas que desempenham tarefas específicas, para que o dado chegue a quem ou a o que o está esperando.

\section{Visão geral de uma aplicação IoT}
\label{section:overview}

Na \ref{fig:1.1.0/iot_app} ilustrada, temos uma rede de N sensores que enviam dados telemétricos e M atuadores que recebem ordens para executar uma função, todos estão em rede e podem receber e enviar informação em tempo real. O servidor, que pode ser um Broker como será descrito adiante neste trabalho, encaminha os dados (ou mensagens) para o banco de dados. O Banco é utilizado para análise dos dados, o controlador por sua vez envia as mensagens de decisões baseada na análise de dados a serem transmitidas para os atuadores.

\begin{figure}[h!]
\centering
\includegraphics[width=10cm]{./02_Capitulos/02_Cap1/figures/iot_app}
\caption{Um exemplo de aplicação IoT que percorre os problemas a solucionar}
\label{fig:1.1.0/iot_app}
\end{figure}

O sensor 1 está imerso em um ambiente com usas próprias características físicas e de rede, isso ocorre com todos, isto é cada sensor está imerso num cenário próprio, variando de redes com poucos sensores a redes com grande fluxo de dados, sujeito a congestionamento. Assim, seria de grande ajuda que o  sistema se ajustasse aos diferentes cenários.

Este trabalho visa implementar um sistema que utiliza o protocolo de aplicação MQTT, para transmissão de dados em tempo real entre dispositivos). O sistema contempla também persistência de dados utilizando banco de dados MongoDB, com o diferencial de se adaptar a cenários através de canais de dados chamados Data Streams, oferecendo diferentes configurações que podem otimizar o envio de dados em cada cenário.

\begin{figure}[h!]
\centering
\includegraphics[width=10cm]{./02_Capitulos/02_Cap1/figures/iot_app-layers}
\caption{A aplicação dividida em blocos exercendo um papel em uma aplicação IoT}
\label{fig:1.1.0/iot_app-layers}
\end{figure}


Cada bloco é responsável por uma tarefa no sistema IoT, da aquisição de dados a persistência destes, conforme ilustrado na \ref{fig:1.1.0/iot_app-layers}. Para entender melhor cada tarefa, será descrito o projeto e as implementações em cada bloco no sistema.

Inicialmente, os conceitos e ideias do projeto eram voltados a desenvolver uma interface no qual um desenvolvedor poderia implementar um sistema IoT de ponta a ponta utilizando APIs que direcionariam para um desses protocolos da seção \ref{section:tecnologias_iot}, porém as diferenças entre os protocolos e as camadas de base, fazem com que esta solução esteja mais distante. Então o foco voltou-se  para tecnologias que tenham base na pilha TCP/IP, por sua vasta implementação nas redes industriais e residências e na Internet.
\section{Interface para Protocolos de Aplicação}
\label{section:interface_iot}

Neste projeto será visto a implementação desta interface para o protocolo MQTT, entretanto a idéia é estender o interfaceamento com outros protocolos com caracterísitcas favoráveis para uma rede IoT, como mostrado na \ref{fig:2.2.0/camada_abatracao}, assim como apresentar uma estrutura que se traduza aos protocolos sobre o TCP/IP. Algumas características fundamentais podem ser destacadas como fundamentais:

\begin{itemize}

\item Full-Duplex. Capaz de receber e enviar mensagens ao mesmo tempo;
\item Multicast. Capaz de enviar mensagens um ou mais dispositivos simultâneos;
\item Envio de mensagens em tempo real;

\end{itemize}


\begin{figure}[h!]
\centering
\includegraphics[width=12cm]{./02_Capitulos/02_Cap2/figures/camada_abstracao}
\caption{Interface de comunicação. A interface tem seu próprio protocolo que direciona e se comunica a um ou mais protocolos de aplicação}
\label{fig:2.2.0/camada_abatracao}
\end{figure}


\chapter{A Interface e sua ligação com IoT}
\label{chapter:interface_iot}

No capítulo \ref{chapter:industria_4_0_iot}, foi visto as bases para se implementar um projeto de IoT. A área começou a receber fortes investimentos e atenção por volta de 2009 e desde então foram feitas consideráveis implementações utilizando tecnologias e protocolos diferentes. Neste capítulo serão apresentados algumas dessas variações, para fins de comparação e respaldo para importância e objetivo deste projeto.


\section{Algumas tecnologias em IoT}
\label{section:tecnologias_iot}

Estas são algumas tecnologias que satisfazem as condições apresentadas para um sistema IoT, nem todas utilizam o protocolo TCP/IP, mas todas são  capazes de fazer seus dispositivos comunicarem-se em tempo real levando em consideração seus alcances e escalabilidade.

\subsection{RFID}
\label{subsection:rfid}

As primeiras aplicações de IoT foram em laboratórios de aplicações de RFID \cite{Rampim:iot}, junto com códigos bidimensionais, para aplicações de identificação de objetos. Uma das soluções mais populares e de baixo custo de IoT utilizando Rádio frequência.

\subsection{Redes NB}
\label{subsection:nb}

Redes que utilizam bandas restritas visando baixo consumo e distância de transmissão são a nova fronteira, as mensagens de IoT são geralmente curtas, dados telemétricos, status etc, logo estes protocolos mostram-se úteis para este tipo de aplicação. Já se encontram implementadas algumas redes como SigFox \cite{Sigfox} e LoRa \cite{LoRa}. 

\subsection{Low Energy Bluetooth}
\label{subsection:bluetooth}

As novas gerações de Bluetooth consomem muito menos energia, o que tornaram a tecnologia viável para aplicações IoT. Geralmente, módulos Bluetooth são utilizados como beacons \cite{Endeavor:Beacons}. Pontos espalhados por uma região, no qual podem se comunicar com os módulos de dispositivos mobile ao se aproximar, oferecendo links para conteúdo e exclusividades.

\subsection{Base TCP/IP}
\label{subsection:tcpip}

As tecnologias mais comuns de se encontrar em aplicações IoT, os protocolos construídos com base no TCP/IP são vastamente utilizados e possuem uma rede mundialmente distribuída, o que facilita o uso. Pode-se implementar uma gama de protocolos de aplicações, alguns mais eficientes que outros.

O protocolo mais simples seria o HTTP, altamente usado na internet, porém não é eficiente no consumo de energia por abrir uma conexão a cada envio de dados. Para minimizar estas desvantagens, foi desenvolvido o CoAP \cite{coap} protocolo nos mesmos moldes do HTTP com o modelo REST, porém mais simples, mais leve, com baixo overhead e utilizado em redes locais.

Mas os mais utilizados em aplicações são sem dúvidas os protocolos que mantém conexão aberta, em especial Websocket e MQTT, sendo o primeiro mais utilizado para chats e mensagens e o segundo domina o mundo do M2M e Telemetria.


\section{A Interface}
\label{section:interface}

Inicialmente, os conceitos e ideias do projeto eram voltados a desenvolver uma interface no qual um desenvolvedor poderia implementar um sistema IoT de ponta a ponta utilizando APIs que direcionariam para um desses protocolos da seção \ref{section:tecnologias_iot}, porém as diferenças entre os protocolos e as camadas de base, fazem com que esta solução esteja mais distante. Então o foco voltou-se  para tecnologias que tenham base na pilha TCP/IP, por sua vasta implementação nas redes industriais e residências e na Internet.

Neste projeto iremos ver a implementação desta interface para o protocolo MQTT, cuja escolha será justificada adiante. Serão descritas as interfaces para as três camadas, que são de baixo custo, open-source e altamente escaláveis para construir outras aplicações com esta como base.
 


\chapter{O Projeto}
\label{chapter:projeto}

Os dois últimos capítulos descreveram o conceito de Internet das Coisas e as especificações que o projeto deve contemplar, visando sempre ajudar na construção de sistemas IoT que melhor se encaixem na aplicação. Neste capítulo serão descritos as implementações do projeto, apresentado os motivos das escolhas de tecnologias e protocolos especificados. E terminando sobre persistência de dados em aplicações IoT e por quê a escolha de implementação bancos de dados é importante para a aplicação.


\section{Camada de Abstração}
\label{section:camada_abstracao}

Devido a interação entre dispositivos de aquisição de dados e aplicação e armazenamento de dados, foi necessário uma implementação de um protocolo de comunicação único entre os dispositivos e implementação em cada um destes em suas diferentes linguagens de programação.

O protocolo consiste em uma abstração de um canal de envio de dados chamado Data Stream mostardo em \ref{fig:3.1.0/data_stream}, no qual passam dados após realizar um processamento dos dados em uma determinada velocidade podendo conter um limite de pacote de dados. Nas pontas desse canal estão os Publishers e Subscribers, que serão descritos adiante. Este conceito é uma forma de abstrair, unificar e simplificar a forma de transporte de dados, de uma modo que a interface possa ter o controle sobre os aspectos de transmissão. Cada protocolo na camada de aplicação, implementa este conceito de uma certa forma, porém o desenvolvedor não precisará se preocupar com estes detalhes.

\begin{figure}[h!]

\centering
\includegraphics[width=13cm]{./02_Capitulos/02_Cap3/figures/data_stream}
\caption{O conceito de Data Stream para a abstração do transporte de dados}
\label{fig:3.1.0/data_stream}
\end{figure}


\section{Publishers e Subscribers}
\label{section:publishers_subscribers}

Para enviar e receber dados de uma forma a atender os requisitos da seção \ref{section:interface}, foi utilizado um padrão de comunicação recorrente em aplicações contemporâneas, o padrão Publish/Subscribe \cite{amazon:pub_sub}.

O padrão Publish/Subscribe permite que as mensagens sejam transmitidas assíncronas e para vários dispositivos simultaneamente. Para transmitir uma mensagem, um client pode simplesmente enviar uma mensagem para o tópico que os envia imediatamente para todos os subscribers. Todos os componentes que se inscreverem no tópico receberão todas as mensagens transmitidas, a menos que uma política de filtragem de mensagens seja definida pelo assinante.

Qualquer mensagem publicada em um tópico é imediatamente recebida por todos os subscribers do tópico. As mensagens de podem ser usadas para arquiteturas orientadas a eventos ou para desacoplar aplicativos, aumentando  o desempenho, a confiabilidade e a escalabilidade. Com isso, foram criados duas funções possíveis para cada dispositivo dentro deste padrão, os Publishers e os Subscribers, sua comunicação é descrita em \ref{fig:3.2.0/pub_sub}.

\begin{figure}[h!]
\label{fig:3.2.0/pub_sub}
\centering
\includegraphics[width=12cm]{./02_Capitulos/02_Cap3/figures/publisher-subscriber_comm}
\caption{Comunicação entre Publishers e Subscribers por Data Stream}
\label{fig:3.2.0/pub_sub}
\end{figure}


Publishers são dispositivos que criam Data Stream  e enviam dados por estes, regulam o processamento dos dados estipulam limites de tamanho de cada pacote de dado e determinam o intervalo de envio de pacotes. O protocolo permite que estes enviem os dados e também permite que outros dispositivos possam passar configurações remotamente para modificar os parâmetros de cada Data Stream, como o intervalo de envio ou outra configuração criada pelo tipo de Data Stream implementado. 

Subscribers estão na outra ponta recebendo os dados, são capazes de enviar as configurações do Data Stream para os Publishers a chegada destes dados como um driver para a aplicação
Essas funcionalidades foram implementadas Orientadas a Objeto e são escaláveis para aplicações mais complexas que serão implementadas para o uso dos sistemas em aplicações de sensoriamento e visualização dos dados.

\section{A implementação}
\label{section:implementacao}

%%% MQTT , Websocket e HTTP %%%


\subsection{MQTT}
\label{subsection:mqtt}

O protocolo MQTT \ref{} foi utilizado escolhido por ser leve e ideal para aplicações em tempo real com vários dispositivos simultaneamente. É um protocolo nos moldes publish/subscribe  ideal para definir a função de cada dispositivo seja enviando dados (Publish) ou recebendo estes (Subscribe).

Para gerenciar os clients (responsáveis pela implementação da comunicação MQTT) em cada dispositivo é necessário um servidor chamado Broker. Este foi implementado com o Mosquitto \ref{}, um broker open source e leve capaz de ser instalado localmente e no servidor do laboratório para testes remotos.
\chapter{Casos de Uso}
\label{chapter:casos_de_uso}

% Escrever sobre caso de uso do sistemas
No capítulo \ref{chapter:projeto} descrevemos o funcionamento da Interface e sua comunicação com a linguagem MQTT. Este capítulo busca demonstrar o funcionamento do sistema em hardwares com a interface implementada pelos softwares descritos na seção \ref{section:codigos_fonte} disponível no Apêndice. São aplicações simples que mostram a facilidade e a escalabilidade do sistema, além de demonstrar como o sistema pode ser implementado em plataformas.

\section{Medição de temperaturas de CPU}
\label{section:temp_cpu}

Este exemplo tem como objetivo medir a temperatura da CPU de um console com baseado em suas atividades, serviços e processos em execução. A aplicação pode ser escalada para a obtenção de outras informações da CPU e do sistema, podendo assim disponibilizar análises de desempenho da plataforma, além de montar perfis de uso do sistema e administrar seu uso.

Para isso precisaremos utilizar um Publisher no console a ter informações de temperatura a ser coletadas e um Subscriber para receber estas temperaturas via MQTT e persisti-las em banco de dados. Ambas as aplicações utilizarão as APIs em Javascript, utilizando Node.js para coletar as informações do sistema, implementar o Publisher, o Subscriber, o driver para MongoDB (também disponível em anexo) e a geração de um gráfico utilizando a plataforma plotly \cite{plotly}.


\begin{figure}[h!]
\centering
\includegraphics[width=11.5cm]{./02_Capitulos/02_Cap4/figures/fluxo_controle_temp}
\caption{Diagrama de fluxos do Publisher e do Subscriber}
\label{fig:4.1.0/fluxo_controle_temp}
\end{figure}

% Falar sobre o diagrama de fluxo
A \ref{fig:4.1.0/fluxo_controle_temp} mostra todo o fluxo das duas aplicações, o Publisher publica em no tópico \textit{/001/stream:periodic}, a informação coletada a cada T1=3 segundos e espera T2=1 antes de enviar. O Subscriber escuta este tópico e persiste ao chegar uma mensagem de dados pelo Data Stream, ao atingir 100 amostras, um gráfico de Temperatura da CPU principal pela Data-Hora de inserção é gerado com as últimas 100 inserções no banco.


% Inserção no banco


% Visualização com o plotly




%\chapter{Estimativas de Custos}
\label{chapter:estimativa}

Para validar o baixo custo do sistema, este capítulo tem o foco em simular o orçamento de dois projetos. Um feito para a indústria e outro em redes domésticas para automações residenciais. Serão propostos cenários de aplicações reais, capazes de confirmar a premissa do projeto.
Como vimos no capítulo de projetos, o sistema pode ser instalado em plataformas mais sofisticadas, contendo sistemas operacionais e hardware dedicado. Mas nestes cenários utilizaremos o hardware mínimo que é compatível com o sistema e atua satisfatoriamente na aplicação.


\section{Aplicação: Automação Residencial}
\label{section:residencial}

O caso em estudo é a automação parcial de uma residência. Como pode ser visto em  \ref{fig:5.1.0/planta-casa}, a casa possui dois quartos, sala de estar, dois banheiros, cozinha, área de serviço e varanda. O objetivo é monitorar a temperatura local, o consumo de energia, detectar aberturas de portas e janelas de entrada da residência para fins de segurança e acionamento de luzes.

Para isso é necessário acionadores para a sala e cozinha, mais os quartos, totalizando cerca de 4 pontos de luz para acionar, o mesmo vale para os sensores de temperatura, no qual farão a média de temperatura da casa. Para controle de consumo de energia, nos limitaremos as tomadas de eletrodomésticos e eletrônicos, que representam a maior parte do consumo, para um ponto na sala, nos quartos, na tomada da geladeira, micro-ondas e lavadora, cerca de 6 tomadas a colocar sensores de tensão e corrente AC para cálculo da potência. Serão colocados sensores magnéticos para detectar abertura de portas e janelas, na porta de entrada, na porta da varanda, nas janelas do quarto e na área de serviço.

\begin{figure}[h!]
\centering
\includegraphics[width=13cm]{./02_Capitulos/02_Cap5/figures/planta-casa}
\caption{Planta baixa de residência de dois quartos, retirado de \cite{decorandocasas}}
\label{fig:5.1.0/planta-casa}
\end{figure}

Com menos de R\$ 2000,00 pode-se instalar um sistema robusto para automatizar uma casa. O projeto já conta que a residência possui uma rede WiFi e se a casa tiver um PC, pode-se abater do custo do servidor.


\begin{table}[h!]
\centering
\caption{Orçamento de um sistema simples para automação da residência da \ref{fig:5.1.0/planta-casa}}
\begin{tabular}{|l|l|l|l|l|}
\hline
Item                & Descrição                    & Qtde & Unidade (R\$) & Total (R\$) \\ \hline
esp32               & Módulo de aquisição          & 4    & \$30.00       & \$120.00    \\ \hline
DHT11               & Sensor de temperatura        & 4    & \$10.00       & \$40.00     \\ \hline
P8                  & Módulo sensor de tensão      & 6    & \$20.00       & \$120.00    \\ \hline
Acs712 - 5a         & Módulo sensor de corrente    & 6    & \$15.00       & \$90.00     \\ \hline
Ssr-25              & Relé Estado Sólido           & 4    & \$30.00       & \$120.00    \\ \hline
Desktop             & Servidor Local               & 1    & \$600.00      & \$600.00    \\ \hline
Sensores Magnéticos & Sensores de abertura         & 6    & \$40.00       & \$240.00    \\ \hline
Infraestrura        & Caixas de proteção, fios etc & 1    & \$500.00      & \$500.00    \\ \hline
\multicolumn{4}{|l|}{TOTAL:}                                              & \$1,830.00  \\ \hline
\end{tabular}
\label{table:planta-casa}
\end{table}




% inserir demais capítulos aqui
% -----------------------------
% -----------------------------
% -----------------------------
% -----------------------------





\pagebreak
\addcontentsline{toc}{chapter}{\hspace{1.7cm}\bfseries CONCLUSÃO}
\noindent\textbf{CONCLUSÃO}
$\!$\\


% Falar sobre o projeto em si
Neste projeto, foram apresentados todos os aspectos de hardware e software em seus capítulos. Encontrou-se uma certa dificuldade em implementar o Data Stream nas duas arquiteturas apresentadas, de modo a ter o mesmo funcionamento em ambas. As linguagens e plataformas usadas possuem uma grande quantidade de bibliotecas que facilitaram a realização dos códigos que implementam Publishers, Subscribers e Data Streams. 

O sistema segue sua proposta de escopo aberto, escalável e de baixo custo. Foram apresentadas plataformas de hardware aberto, oferecendo a possibilidade de uma organização ou empresa distribuírem suas próprias versões e baratear ainda mais os custos em larga escala. Os softwares são de escopo aberto e licenças permissíveis, o que significa que possuem versões gratuitas e escaláveis, de modo a também oferecerem a possibilidade de criar versões personalizadas.

O projeto englobou conceitos nas áreas de Eletrônica, Telecomunicação e Computação espalhadas pelas camadas de IoT. O aprendizado e a integração de ferramentas nessas áreas, permitiu criar um caso de uso que validou o funcionamento do sistema. A medição de temperatura das CPUs foi uma forma de observar o comportamento do sistema, conforme o aumento do número de dispositivos utilizados, além de gerar conclusões a partir das medições de temperatura. Porém, o diferencial são os Data Streams. Sua implementação permitiu diferentes formas de transmitir dados em uma mesma aplicação de uma forma que a arquitetura Publish/Subscribe não permite, além de mudanças dinâmicas destas formas de transmissão.

Planeja-se dar um maior enfoque na implementação dos Data Streams em versões futuras, criando novos tipos de Data Streams e expandindo o suporte para outras linguagens de programação. O sistema já é compatível com o protocolo de segurança SSL/TLS, porém outras formas de segurança como encriptação das mensagens e suporte para Blockchain estão no planejamento para novas versões do sistema. Por fim, planeje-se implementar futuramente as funcionalidades de adicionar mais de um broker, para aplicações que exigem um grande fluxo de dados e estender a interface para outros protocolos de aplicação.


\pagebreak





\pagebreak
\addcontentsline{toc}{chapter}{\hspace{1.7cm}\bfseries REFERÊNCIAS}
\def\bibname{REFERÊNCIAS}


%\nocite{*}
% abaixo segue a chamada para o arquivo [.BIB]. Utilizei o programa JABREF para montar o arquivo com minhas referências.
\bibliography{bib/dissertacao_folim}

\definecolor{light-gray}{gray}{0.97}


\chapter{Apêndice}
\label{chapter:apendice}

\section{Guias de instalação}
\label{section:guia}

\textbf{Atenção:} Essas instruções são uma versão de quando este trabalho foi publicado, para obter a versão mais atualizada e outras versões, acesse \url{https://github.com/fol21}. Este guia tem como reproduzir os testes feitos no sistema, que foi testado em distribuições de Windows 10 e Linux.

\subsection{Configurando Broker}
\label{subsection:guia_broker}

\begin{enumerate}

\item Acesse \url{https://mosquitto.org/download/} e siga as instruções para baixar e instalar o Mosquitto;
\item No terminal execute o comando \textit{mosquitto -d}
\item Use o IP e a porta configurada nos exemplos de Publisher e Subscriber;
\item Em C++ as configurações são encontradas no próprio código;
\item Em Javascript acesse o arquivo \textit{config.json} no diretório \textit{example/*/resources} e mude as configurações de IP e Porta;
\end{enumerate}

\subsection{Publishers em C++}
\label{subsection:guia_publishers_cpp}

\begin{enumerate}

\item Siga as instruções de instalação em \url{http://docs.platformio.org/en/latest/installation.html};
\item Acesse \url{https://docs.espressif.com/projects/esp-idf/en/latest/get-started/establish-serial-connection.html} e instale os drivers adequados para o ESP32;
\item Acesse \url{https://github.com/fol21/things-4-labs-acquirer-platform} e baixe o projeto;
\item No diretório do projeto digite os comandos \textit{pio run -t upload -e esp32};
\end{enumerate}

\subsection{Publishers em Javascript}
\label{subsection:guia_publishers_javascript}

\begin{enumerate}
\item Acesse \url{https://nodejs.org/en/} e baixe sua versão do Node.js na versão 8 ou maior;
\item Acesse \url{https://github.com/fol21/things-4-labs-acquirer-raspberrypi} e baixe o projeto;
\item Entre no diretório examples/gpio, presente no projeto;
\item Coloque as informações de IP e Porta no arquivo \textit{resources/config.json};
\item Execute o comando como administrador \textit{npm install};
\item Execute o comando como administrador \textit{node index.js};

\end{enumerate}


\subsection{Subscribers em Javascript}
\label{subsection:guia_subscribers_javascript}

\begin{enumerate}
\item Acesse \url{https://nodejs.org/en/} e baixe sua versão do Node.js na versão 8 ou maior;
\item Acesse \url{https://docs.mongodb.com/manual/installation/} e siga as instruções para instalar e iniciar o mongodb conforme seu Sistema Operacional;
\item Crie uma conta gratuita em \url{https://plot.ly/};
\item Acesse \url{https://github.com/fol21/things-4-labs-console-subscriber} e baixe o projeto;
\item Entre no diretório examples/simple-subscriber, presente no projeto;
\item Coloque as informações de IP e Porta no arquivo \textit{resources/config.json};
\item Coloque seu Id e chave da sua conta do Plotly em \textit{index.js};
\item Execute o comando como administrador \textit{npm install};
\item Execute o comando como administrador \textit{node index.js};

\end{enumerate}


\section{Códigos Fonte}
\label{section:codigos_fonte}

\textbf{Atenção:} Estes códigos são uma versão de quando este trabalho foi publicado, para obter a versão mais atualizada e outras versões, acesse \url{https://github.com/fol21}.

\subsection{Publishers em C++}
\label{subsection:publishers_cpp}


\lstinputlisting[language=C++,backgroundcolor= \color{light-gray}, caption=Data Stream Header, breaklines=true,basicstyle=\footnotesize\ttfamily]{./03_Conclusao/src/publisher-c++/data_stream.h}

\lstinputlisting[language=C++,backgroundcolor= \color{light-gray}, caption=Data Stream Source, breaklines=true,basicstyle=\footnotesize\ttfamily]{./03_Conclusao/src/publisher-c++/data_stream.cpp}

\lstinputlisting[language=C++,backgroundcolor= \color{light-gray}, caption=MQTT Publisher Header em C++, breaklines=true,basicstyle=\footnotesize\ttfamily]{./03_Conclusao/src/publisher-c++/MqttPublisher.h}

\lstinputlisting[language=C++,backgroundcolor= \color{light-gray}, caption=MQTT Publisher Source em C++, breaklines=true,basicstyle=\footnotesize\ttfamily]{./03_Conclusao/src/publisher-c++/MqttPublisher.cpp}



\subsection{Publishers em Javascript}
\label{subsection:publishers_javascript}


\lstinputlisting[language=C++,backgroundcolor= \color{light-gray}, caption=Data Stream em javascript, breaklines=true,basicstyle=\footnotesize\ttfamily]{./03_Conclusao/src/publisher-js/DataStream.js}

\lstinputlisting[language=C++,backgroundcolor= \color{light-gray}, caption=Continous Stream, breaklines=true,basicstyle=\footnotesize\ttfamily]{./03_Conclusao/src/publisher-js/ContinousStream.js}

\lstinputlisting[language=C++,backgroundcolor= \color{light-gray}, caption=Periodic Stream em Javascript, breaklines=true,basicstyle=\footnotesize\ttfamily]{./03_Conclusao/src/publisher-js/PeriodicStream.js}

\lstinputlisting[language=C++,backgroundcolor= \color{light-gray}, caption=MQTT Publisher Source em Javascript, breaklines=true,basicstyle=\footnotesize\ttfamily]{./03_Conclusao/src/publisher-js/MqttPublisher.js}

\lstinputlisting[language=C++,backgroundcolor= \color{light-gray}, caption=Index das implementações, breaklines=true,basicstyle=\footnotesize\ttfamily]{./03_Conclusao/src/publisher-js/index.js}


\subsection{Subscribers em Javascript}
\label{subsection:subscribers_javascript}

\lstinputlisting[language=C++,backgroundcolor= \color{light-gray}, caption=MQTT Subscriber Source em Javascript, breaklines=true,basicstyle=\footnotesize\ttfamily]{./03_Conclusao/src/subscriber-js/MqttSubscriber.js}

\lstinputlisting[language=C++,backgroundcolor= \color{light-gray}, caption=Index das implementações, breaklines=true,basicstyle=\footnotesize\ttfamily]{./03_Conclusao/src/subscriber-js/index.js}

\lstinputlisting[language=C++,backgroundcolor= \color{light-gray}, caption=Data Client para MongoDB, breaklines=true,basicstyle=\footnotesize\ttfamily]{./03_Conclusao/src/subscriber-js/MongoDataClient.js}

\subsection{Códigos fonte das aplicações em consoles}
\label{subsection:teste_fonte}


\lstinputlisting[language=C++,backgroundcolor= \color{light-gray}, caption= Teste do publisher, breaklines=true,basicstyle=\footnotesize\ttfamily]{./03_Conclusao/src/publisher-js/example.js}

\lstinputlisting[language=C++,backgroundcolor= \color{light-gray}, caption=Teste do subscriber, breaklines=true,basicstyle=\footnotesize\ttfamily]{./03_Conclusao/src/subscriber-js/example.js}

\subsection{Códigos fonte das aplicações em plataformas embarcadas}
\label{subsection:teste-plataformas}


\lstinputlisting[language=C++,backgroundcolor= \color{light-gray}, caption= Teste do publisher no ESP32, breaklines=true,basicstyle=\footnotesize\ttfamily]{./03_Conclusao/src/publisher-c++/main.cpp}




% \printindex    %Removi o índice remissivo para a versão oficial do trabalho.


\end{document}